\newpage
\chapter{Работа в основния прозорец}
\label{chapter02}

На различните операционни системи графичният потребителски интерфейс\index{потребителски интерфейс} изглежда по леко различен начин, но основните функционалности са едни и същи (Фиг. \ref{figure0007}).

\section{Основен работен екран}

\begin{figure}[h!]
  \centering
  \includegraphics[width=1.0\linewidth]{pic0007}
  \caption{Заглавна лента с менюта}
\label{figure0007}
\end{figure}
\FloatBarrier

Най-отгоре (Фиг. \ref{figure0007} - 1), стои заглавната лента\index{заглавна лента} в която е изписано името на файла с който се работи. Когато не е зареден вече съществуващ файл, за име на файл се използва служебна комбинация от букви и цифри. В лявата част на заглавната лента се намират бутоните за управление на прозореца\index{управление на прозореца} (Фиг. \ref{figure0007} – 2). Над заглавната лента се намира лентата с менютата\index{лента с менюта} (Фиг. \ref{figure0007} – 3). На различните операционни системи лентата с менютата може да стои на различно място спрямо заглавната лента. Под заглавната лента стои панелът с инструментите (Фиг. \ref{figure0007} – 4). Под панела с инструментите стои панелът за въвеждане на формули\index{въвеждане на формули} (Фиг. \ref{figure0007} – 5). От ляво на панела за формули стои панела за въвеждане на адрес, за клетка или регион от клетки\index{адресиране на клетки} (Фиг. \ref{figure0007} – 6). Долу, в ляво са разположение работните листи\index{работни листи} (Фиг. \ref{figure0007} – 7). От дясно на работните листи се намира лентата за състоянието\index{лента на състоянието} на документа (Фиг. \ref{figure0007} – 8). Хоризонталният (Фиг. \ref{figure0007} – 9) и вертикалният (Фиг. \ref{figure0007} – 10) плъзгачи служат за преглеждане на работната област, когато тя надвишава видимата област от екрана. 

\section{Менюта}

Менютата\index{менюта} на програмния продукт съдържат команди\index{команди}, които са групирани според действията за които са предназначени (Фиг. \ref{figure0008}). В менюто OpenOffice са организирани командите, които се отнасят до функционалности в целия пакет, а не само в модула Calc. В менюто File са разположени команди за работа с файлове (зареждане, съхраняване, експортиране и други). Менюто Edit съдържа команди за редактиране на информация, като копиране, поставяне и други. Менюто View дава различни възможности за визуализация на активния документ. Менюто Insert дава възможности за вмъкване на различни обекти в работния документ. Менюто Format дава възможности за форматиране на различни обекти или части от документа. В менюто Tools са поместени инструменти за по-специфична обработка, като речници за езика, модули за проверка на граматиката и правописа, команди за работа с макроси и други. В менюто Data са поместени команди за обработка на данните, като сортиране или филтриране. Менюто Window дава възможности за работа с активните прозорци. И последното меню Help служи за допълнителна информация относно цялостната работа на системата. 

\begin{figure}[h!]
  \centering
  \includegraphics[width=1.0\linewidth]{pic0008}
  \caption{Главна група менюта}
\label{figure0008}
\end{figure}
\FloatBarrier

За да се изпълни команда от меню системата, то първо с мишката трябва да се избере конкретно меню, а след това опция в самото меню. Не всички команди по менютата са винаги достъпни. Това е така, защото някои от командите са контекстно зависими и имат смисъл само при определени условия. Недостъпните команди са изобразени в бледи цветове. С цел ускоряването на работата с програмния продукт, някои от командите в меню системата позволяват извикване с бърза клавишна комбинация. Разбира се, тези клавишни комбинации може да се различават в определена степен на различните операционни системи. Някои команди в меню системата от своя страна също представляват менюта\index{под менюта} и са обозначени със стрелка от дясно на названието им. При избора на тези команди се отваря допълнително меню, което на свой ред дава възможност за избор от списък команди (Фиг. \ref{figure0009}). 

\begin{figure}[h!]
  \centering
  \includegraphics[width=1.0\linewidth]{pic0009}
  \caption{Под меню в основно меню}
\label{figure0009}
\end{figure}
\FloatBarrier

\section{Палитра с инструменти}

Палитрите с инструменти са организирани под лентата с менютата и също са групирани в панели. Палитрите с инструменти съдържат бутони за бързо изпълнение на най-често използваните команди. Както менютата, така и палитрите с инструменти подлежат на настройване от опциите на програмния продукт. Тези възможности дават свобода на потребителя да организира максимално ефективно виртуалното си работно пространство. Някои палитри с инструменти не са често използвани (примерно палитрата с инструменти за рисуване) и поради тази причина не са видими в основния екран. Палитрите, които не са видими биват активирани, когато се работи с клетки или обекти за които са предназначени. Повечето палитри с инструменти са прикачени за някои от ръбовете на основния прозорец, но е възможно палитрите да бъдат откачени и да имат свой самостоятелен мини-прозорец. 

\begin{figure}[h!]
  \centering
  \includegraphics[width=1.0\linewidth]{pic0010}
  \caption{Свободно плаваща палитра с инструменти}
\label{figure0010}
\end{figure}
\FloatBarrier

