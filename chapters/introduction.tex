\newpage
\addcontentsline{toc}{chapter}{Предговор}
\chapter*{Предговор}
\pagenumbering{arabic}
\setcounter{page}{1}
\pagestyle{fancyplain}

Calc е софтуер за електронни таблици, който влиза в състава на програмния продукт Apache OpenOffice. По настояще Calc се разпространява като софтуер с отворен код под два основни лиценза – SISSL и GNU LGPL. Calc основно се използва за създаване и оформление на таблици, изчисления, обработка и анализ на данни, както и за графично представяне на резултатите от обработката. Учебното помагало е предназначено за офис работници (ръководители, мениджъри, секретари, финансови анализатори и други), ученици и студенти. 

Учебното помагало представя графичния интерфейс на Calc. Демонстрирате се основните елементи на потребителския интерфейс и начина за работа с тях. Засягат се темите за работа с файловата система на операционната система. Отделя се внимание за начина по който Calc зарежда и съхранява файлове, както и за възможностите да обработва информация от алтернативни софтуерни продукти за електронни таблици. Засягат се въпросите за работа с документи и списъчна информация. Набляга се на въвеждането, манипулирането и обработването на таблична информация. Значително внимание се отделя на възможностите за изчисления в Calc. Акцент е работата с формули, организиране на изчисленията и използването на наличните в Calc математически функции. Допълнително внимание е отделено за форматирането на таблиците и графичното им оформление. Разглеждат се възможностите за сортиране и филтриране на данни. Демонстрират се основни принципи за защита на информацията. Представя се работа с диаграми от различни видове – създаване, редакция, оформление и настройки. Засягат се темите за оформление на печатните страници и разпечатването на принтер. 

Глава 1 - \nameref{chapter01}: Представя процеса по изтегляне, инсталиране и стартиране на програмния продукт.

Глава 2 - \nameref{chapter02}: Запознава с основните елементи на графичния потребителски интерфейс в основния работен прозорец.

