\documentclass[runningheads,14pt,a4paper,openany]{book}

% Добавя възможност за сензитивни хипер-връзки в самия документ.
\usepackage[linktocpage=true,bookmarks=false]{hyperref}
\usepackage{nameref}
\usepackage[utf8x]{inputenc}
\usepackage[english,bulgarian]{babel}
\usepackage{url}
\usepackage{shorttoc}
\usepackage[pdftex]{graphicx}
\usepackage{imakeidx}
\usepackage{placeins}
% Използва се за включване на кориците под формата на PDF файлове.
\usepackage{pdfpages}
% Служи за управление на заглавията.
\usepackage{fancyheadings}

% Добавени от доц. Вера Ангелова.
%\usepackage{amsmath,amssymb,amsthm}
%\usepackage{longtable}
%\usepackage{pifont}   
%\usepackage{epsfig}
%\usepackage{tocloft}
%\usepackage{etoc}
%\usepackage{wasysym}
%\usepackage{eurosym}
%\usepackage{slashbox}
%\usepackage{soul}
%\usepackage{enumitem}
%\usepackage{mathcomp}
%\usepackage{epstopdf}
%\usepackage{latexsym}
%\usepackage{eucal}
%\usepackage{mathrsfs}

% Директория в която се намират изображенията.
\graphicspath{{images/}}

\textheight 22.8cm
\textwidth 17cm
\oddsidemargin -0.54cm
\evensidemargin -0.54cm
\topmargin 1.5cm

\parskip=0.2cm
\parindent=20pt
\flushbottom

% Премахва подчертаващата линия в заглавните части.
\renewcommand{\headrulewidth}{0pt}

\selectlanguage{bulgarian}

\lhead[\thepage \quad Тодор Балабанов \quad \hfill]{}
\chead{}
\rhead[]{\hfill Apache OpenOffice Calc за корпоративна употреба \quad \thepage }
\lfoot{}
\cfoot[\em Лекции по компютърни науки и технологии на ИИКТ - БАН, № *, 20**]{\em Лекции по компютърни науки и технологии на ИИКТ - БАН, № *, 20**} 
\rfoot{}

\onecolumn
\makeindex[columns=1, title=Азбучен указател, intoc]

\begin{document}

\def\ql{\textquotedblleft}\def\qr{\textquotedblright}

\includepdf[pages={1,2}]{images/front}
\thispagestyle{empty}

\voffset =-1truecm

% Използва се за номерация на страниците.
\renewcommand{\thepage}{\roman{page}}

\setcounter{page}{-1}
\thispagestyle{empty}
\pagestyle{empty}
\thispagestyle{empty}

% Тук стои таблицата със съдържанието, което се генерира от названието на главите.
\newpage
\thispagestyle{empty}
\pagestyle{empty}
\shorttoc{Теми}{0}
\thispagestyle{empty}
\pagestyle{empty}

% Тук стои таблицата със съдържанието, което се генерира от названието на главите и названието на секциите в тях.
\newpage
\thispagestyle{empty}
\pagestyle{empty}
\thispagestyle{empty}
\tableofcontents
\thispagestyle{empty}
\pagestyle{empty}

% Списък с фигурите.
\newpage
\listoffigures
\addcontentsline{toc}{chapter}{Списък на фигурите}

% Списък с таблиците. 
\newpage
\listoftables
\addcontentsline{toc}{chapter}{Списък с таблиците}

\newpage
\addcontentsline{toc}{chapter}{Предговор}
\chapter*{Предговор}
\pagenumbering{arabic}
\setcounter{page}{1}
\pagestyle{fancyplain}

Calc е софтуер за електронни таблици, който влиза в състава на програмния продукт Apache OpenOffice. По настояще Calc се разпространява като софтуер с отворен код под два основни лиценза – SISSL и GNU LGPL. Calc основно се използва за създаване и оформление на таблици, изчисления, обработка и анализ на данни, както и за графично представяне на резултатите от обработката. Учебното помагало е предназначено за офис работници (ръководители, мениджъри, секретари, финансови анализатори и други), ученици и студенти. 

Учебното помагало представя графичния интерфейс на Calc. Демонстрирате се основните елементи на потребителския интерфейс и начина за работа с тях. Засягат се темите за работа с файловата система на операционната система. Отделя се внимание за начина по който Calc зарежда и съхранява файлове, както и за възможностите да обработва информация от алтернативни софтуерни продукти за електронни таблици. Засягат се въпросите за работа с документи и списъчна информация. Набляга се на въвеждането, манипулирането и обработването на таблична информация. Значително внимание се отделя на възможностите за изчисления в Calc. Акцент е работата с формули, организиране на изчисленията и използването на наличните в Calc математически функции. Допълнително внимание е отделено за форматирането на таблиците и графичното им оформление. Разглеждат се възможностите за сортиране и филтриране на данни. Демонстрират се основни принципи за защита на информацията. Представя се работа с диаграми от различни видове – създаване, редакция, оформление и настройки. Засягат се темите за оформление на печатните страници и разпечатването на принтер. 

Глава 1 - \nameref{chapter01}: Представя процеса по изтегляне, инсталиране и стартиране на програмния продукт.

\newpage
\chapter{Инсталация и стартиране}
\label{chapter01}

Софтуерният продукт Calc е част от пакета Apache OpenOffice. Пакетът първоначално е известен с названието StarOffice и е разработен като комерсиален софтуер от немската компания Star Division (от 1985 година). През 1999 година компанията Sun Microsystems изкупува Star Division и през 2000 година Sun публикуват кода на продукта, под отворен лиценз и с променено име – OpenOffice.org. Целта на компанията е да предложи алтернатива на комерсиалният Microsoft Office пакет. През 2010 Oracle Corporation изкупува Sun Microsystems, но продуктът не получава очакваното внимание. Поради стриктно комерсиалната линия водена от Oracle, продуктът е клониран под названието LibreOffice през 2011 година. През същата година Oracle променят лиценза на основния продукт и го предлагат през Apache Software Foundation. 

\section{Сваляне и инсталация}

Продуктът е достъпен от основната си уеб страница на URL адрес (Фиг. \ref{figure0001}): https://www.openoffice.org/download/

\begin{figure}[h!]
  \centering
  \includegraphics[width=1.0\linewidth]{pic0001}
  \caption{Страница за сваляне на OpenOffice}
\label{figure0001}
\end{figure}
\FloatBarrier

На страницата за сваляне\index{сваляне на продукта} на продукта потребителят трябва да избере вида на операционната си система, основният език на който ще бъде потребителския интерфейс и версията, която желае да инсталира на локалната си машина. 

\begin{figure}[h!]
  \centering
  \includegraphics[width=1.0\linewidth]{pic0002}
  \caption{Съгласие с общите условия на SourceForge}
\label{figure0002}
\end{figure}
\FloatBarrier

Следва уеб страница за съгласие с общите условия SourceForge (Фиг. \ref{figure0002}), където се съхраняват бинарните инсталационни файлове на продукта.

\begin{figure}[h!]
  \centering
  \includegraphics[width=1.0\linewidth]{pic0003}
  \caption{Запазване на инсталационния файл}
\label{figure0003}
\end{figure}
\FloatBarrier

След 5 секунди изчакване се появява диалогова кутия, която предлага възможност за съхраняване на инсталационния файл (Фиг. \ref{figure0003}).

\begin{figure}[h!]
  \centering
  \includegraphics[width=1.0\linewidth]{pic0004}
  \caption{Стартиране на инсталатора}
\label{figure0004}
\end{figure}
\FloatBarrier

Двойно кликване с левия бутон на мишката върху файла на инсталатора води до неговото стартиране (Фиг. \ref{figure0004}).

\begin{figure}[h!]
  \centering
  \includegraphics[width=1.0\linewidth]{pic0005}
  \caption{Процес за инсталация}
\label{figure0005}
\end{figure}
\FloatBarrier

Инсталацията\index{инсталация на продукта} основно протича с копиране на файловете в избраната директория за приложения (Фиг. \ref{figure0005}).

\section{Стартиране}

\begin{figure}[h!]
  \centering
  \includegraphics[width=1.0\linewidth]{pic0006}
  \caption{Основен прозорец на Apache OpenOffice Calc}
\label{figure0006}
\end{figure}
\FloatBarrier

След успешна инсталация, през икона за OpenOffice, след стартиране\index{стартиране на продукта} на Calc основният прозорец има изгледът от Фиг. \ref{figure0006}.

Работата с OpenOffice има следните предимства, спрямо алтернативни софтуерни продукти:

* Може да се използва без заплащане на лицензионна такса;

* Може да се използва на всички популярни операционни системи (Windows, Mac OS X, Linux); 

* Интерфейсът е достъпен на повече от 40 езика, като за повече от 70 езика има инструменти за проверка на правописа, сричкопренасяне и речници с думи;

* Поддържат се множество файлови формати, като е възможно експортиране в PDF и работа с Microsoft Office файловите формати. 


\newpage
\chapter{Работа в основния прозорец}
\label{chapter02}

На различните операционни системи графичният потребителски интерфейс\index{потребителски интерфейс} изглежда по леко различен начин, но основните функционалности са едни и същи (Фиг. \ref{figure0007}).

\section{Основен работен екран}

\begin{figure}[h!]
  \centering
  \includegraphics[width=1.0\linewidth]{pic0007}
  \caption{Заглавна лента с менюта}
\label{figure0007}
\end{figure}
\FloatBarrier

Най-отгоре (Фиг. \ref{figure0007} - 1), стои заглавната лента\index{заглавна лента} в която е изписано името на файла с който се работи. Когато не е зареден вече съществуващ файл, за име на файл се използва служебна комбинация от букви и цифри. В лявата част на заглавната лента се намират бутоните за управление на прозореца\index{управление на прозореца} (Фиг. \ref{figure0007} – 2). Над заглавната лента се намира лентата с менютата\index{лента с менюта} (Фиг. \ref{figure0007} – 3). На различните операционни системи лентата с менютата може да стои на различно място спрямо заглавната лента. Под заглавната лента стои панелът с инструментите (Фиг. \ref{figure0007} – 4). Под панела с инструментите стои панелът за въвеждане на формули\index{въвеждане на формули} (Фиг. \ref{figure0007} – 5). От ляво на панела за формули стои панела за въвеждане на адрес, за клетка или регион от клетки\index{адресиране на клетки} (Фиг. \ref{figure0007} – 6). Долу, в ляво са разположение работните листи\index{работни листи} (Фиг. \ref{figure0007} – 7). От дясно на работните листи се намира лентата за състоянието\index{лента на състоянието} на документа (Фиг. \ref{figure0007} – 8). Хоризонталният (Фиг. \ref{figure0007} – 9) и вертикалният (Фиг. \ref{figure0007} – 10) плъзгачи служат за преглеждане на работната област, когато тя надвишава видимата област от екрана. 

\section{Менюта}


\include{chapters/conclusion}

% Списък с използвана литература и източници на информация.
\newpage
\begin{thebibliography}{99}
\addcontentsline{toc}{chapter}{Библиография}
\end{thebibliography}

% Азбучен указател на използваните термини.
\newpage
\printindex

\includepdf[pages=-,height=320mm]{images/back}
\end{document}
